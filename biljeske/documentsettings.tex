%%%%%%%%%%%%%%%%%%%%%%%%%%%%%%%%%%%%%%%%%%%%%%%%%%%%%%%%%%%%%%%%%%%%%%%%%%%%%%%%%%%%%%%%%%%%%%%%%%%%%%%%%%%%%%%%%%%%%%%
%%%%%%%%%%%%%%%%%%%%%%%%%%%%%%%%%%%%%%%%%%%%%%%%%%%%%%%%%%%%%%%%%%%%%%%%%%%%%%%%%%%%%%%%%%%%%%%%%%%%%%%%%%%%%%%%%%%%%%%
%%%%%%%%%%%%%%%%                                                                                          %%%%%%%%%%%%%
%%%%%%%%%%%%%%%% Settingsi su na dnu, tipa margine i tako sto, al ostalo se isto moze urediti             %%%%%%%%%%%%%
%%%%%%%%%%%%%%%%                                                                                          %%%%%%%%%%%%%
%%%%%%%%%%%%%%%%%%%%%%%%%%%%%%%%%%%%%%%%%%%%%%%%%%%%%%%%%%%%%%%%%%%%%%%%%%%%%%%%%%%%%%%%%%%%%%%%%%%%%%%%%%%%%%%%%%%%%%%
%%%%%%%%%%%%%%%%%%%%%%%%%%%%%%%%%%%%%%%%%%%%%%%%%%%%%%%%%%%%%%%%%%%%%%%%%%%%%%%%%%%%%%%%%%%%%%%%%%%%%%%%%%%%%%%%%%%%%%%


%======================================================================================================================
%------------------------- Ovdje kopiram stvari koje su mi bile iz .sty filea, ali zaprvo -----------------------------
%------------------------- se moze vjerojatno maknuti svasta od toga --------------------------------------------------
%----------------------------------------------------------------------------------------------------------------------

\usepackage[croatian]{babel}
%\usepackage[dvipsnames]{xcolor}
\usepackage{amsmath}
\usepackage{amssymb}
\usepackage{wasysym}
\usepackage{textcomp}
\usepackage{array}
\usepackage{arcs}
\usepackage{polynom}
\usepackage{cancel}
\usepackage{enumerate}
\usepackage{dcolumn}
\usepackage{xlop}
\usepackage{geometry}
\usepackage{xcolor}

\usepackage{amsthm}

% OVO JE IZ STILA ZA DIPLOMSKI KOJI ODREDJUJE MARGINE I TAKO TO
%\geometry{bindingoffset=12mm,nomarginpar,includeall,left=23mm,right=23mm,top=35mm,bottom=38mm}
%\setlength\baselineskip{15pt}

%----------------------------------------------------------------------------------------------------------------------
%PAKETI KOJE SAM JA DODAO KOJI NISU STAJALI U ORIGINALNOM PREDLOŠKU ZA DIPLOMSKI

\usepackage{lipsum}
\usepackage{enumitem}				%za enumeraciju itemiziranje
\usepackage{soul}					%za podcrtavanje
\usepackage{soulutf8}				%za podrcrtavanje
\usepackage{centernot}				%za negacjiu implikacije \centernot\implies da prekrizeno bude na crti
\usepackage{abraces}				%za lijepši overbrace i underbrace
\usepackage{import}                 %za lijepo importanje figuresa iz inkscapea

%----------------------------------------------------------------------------------------------------------------------
%======================================================================================================================


%======================================================================================================================
%------------------------- Ovdje definiramo kako nam izlgeda naslovna stranica za \makepage ---------------------------
%----------------------------------------------------------------------------------------------------------------------
%Titlestyle
\pretitle{\begin{center}\Huge}
\posttitle{\end{center}}
\preauthor{\begin{center}\LARGE}
\postauthor{\end{center}}
%----------------------------------------------------------------------------------------------------------------------
%TITLE SPACING
%\setlength{\droptitle}{0pt}
\newlength{\dropBeforeTitle}
%\setlength{\dropBeforeTitle}{30mm}
\newlength{\dropAfterTitle}
%\setlength{\dropAfterTitle}{10mm}
%----------------------------------------------------------------------------------------------------------------------
%----------------------------------------------------------------------------------------------------------------------
%new title commands
\newcommand{\beforetitle}[2]{%
\gdef\beforeTitle{#1\\\vspace*{10pt}\hphantom{\vphantom{A}}#2}} %ovu liniju koda sam samostalno promijenio da stavim dva argumenta za dvije ustanove, i phantom ide da bez obzira na sve taj drugi redak ima zauzet prostor
\newcommand{\beforeTitle}{}
\renewcommand{\maketitlehooka}{%
%\par\noindent%
\begin{center}%
    {\Large\beforeTitle}%
\end{center}%
\vspace*{\dropBeforeTitle}
}
%
\newcommand{\aftertitle}[1]{%
\gdef\afterTitle{#1}}
\newcommand{\afterTitle}{}
\renewcommand{\maketitlehookb}{%
\begin{center}%
    \textsc{\Large\afterTitle}%
\end{center}%
\vspace*{\dropAfterTitle}
}
%
\renewcommand{\maketitlehookc}{%
\vfill
}

%Sada definiramo commandu koja poziva maketitle u titlingpage obliku, tako da naslovna nije numerirana
\newcommand{\makeMyTitle}{%
\begin{titlingpage}%
    \maketitle%
\end{titlingpage}%
}
%----------------------------------------------------------------------------------------------------------------------
%======================================================================================================================


%======================================================================================================================
%------------------------- Ovdje namjestam stil za definiciju, teorem, propozicije i sve to ... -----------------------
%----------------------------------------------------------------------------------------------------------------------

%----------------------------------------------------------------------------------------------------------------------
%----------------OVAJ JEDAN REDAK ISPOD SAM JA DODAO JER MI JE TAKO LJEPŠE RADITI DIPLOMSKI, MAKNUTI PO ŽELJI----------
\theoremstyle{definition}
%----------------------------------------------------------------------------------------------------------------------
%----------------------------------------------------------------------------------------------------------------------
%----------------------------------------------------------------------------------------------------------------------

%\newtheorem{theorem}{Teorem}[\enumeration] % ovo je neki stari pokusaj, ali nije proslo

%ako zelimo enumeraciju da nije 0.0.1 (kada stavis zadatak prije chaptera),
%onda treba u documentsettings.tex zakomentirati
%liniju \newtheorem{theorem}{Teorem}[section]
%i odkomentirati prikladnu liniju
%\newtheorem{theorem}{Teorem}
%\newtheorem{theorem}{Teorem}[chapter]
%\newtheorem{theorem}{Teorem}[section]

%ovo je sada novo s \enumeration commandom u
\enumeration
\newtheorem{lemma}[theorem]{Lema}
\newtheorem{corollary}[theorem]{Korolar}
\newtheorem{definition}[theorem]{Definicija}
\newtheorem{remark}[theorem]{Napomena}
\newtheorem{proposition}[theorem]{Propozicija}
\newtheorem{example}[theorem]{Primjer}
\newtheorem{exercise}[theorem]{Zadatak}

%----------------------------------------------------------------------------------------------------------------------
%======================================================================================================================


%======================================================================================================================
%------------------------- Ovdje sam iskopirao kod s stackexchangea koji mi uredjuje proof style ----------------------
%----------------------------------------------------------------------------------------------------------------------
%-------------------------------------------------------------------
%----------------------REDEF PROOF ENVIRONMENT----------------------
%-------------------------------------------------------------------
%ove komande mjenjaju proof environment u onakav kakav se meni više sviđa, po volji maknuti

\expandafter\let\expandafter\oldproof\csname\string\proof\endcsname
\let\oldendproof\endproof
\renewenvironment{proof}[1][dokaz]{%
  \oldproof[\underline{\textsc{#1}}\nopunct]%
}{\oldendproof}
%-------------------------------------------------------------------
%------------------------ALTERNATIVNO-------------------------------
%-------------------------------------------------------------------
%\newenvironment{myproof}[1][dokaz]{%
%  \proof[\underline{\textsc{#1}}\nopunct]%
%}{\endproof}
%-------------------------------------------------------------------
%-------------------------------------------------------------------
%-------------------------------------------------------------------
%----------------------------------------------------------------------------------------------------------------------
%======================================================================================================================


%======================================================================================================================
%----------------------------------------------------------------------------------------------------------------------
%\hyphenation{di-men-zi-o-nal-nom} NOTE: ovo ne radi iz nekog razloga

%\addto\croatian{\renewcommand\proofname{ABC}}
%\renewcommand*{\proofname}{\textsc{dokaz}}

%\usepackage{ulem}
%mijenja \emph u \underline sa line-breakom
%alternativno, koristiti soul package
%----------------------------------------------------------------------------------------------------------------------
%======================================================================================================================


%======================================================================================================================
%------------------------- Ovdje sam onda isto tako sklepao nesto za solution environment -----------------------------
%----------------------------------------------------------------------------------------------------------------------
%----------------------------------------------------------------------------------------------------------------------
%----------------------------------------------------------------------------------------------------------------------
%----------------------------------------------------------------------------------------------------------------------

%---------DEFINICIJA SOLUTION ENVIRONMENTA--------------------
\newenvironment{solution}[1][rje\v senje]{%
  \oldproof[\underline{\textsc{#1}}\nopunct]%
}{\oldendproof}

%----------------------------------------------------------------------------------------------------------------------
%======================================================================================================================


%======================================================================================================================
%------------------------- Ovdje sam iskopirao niz komandi s stackexchangea koji popravljaju \impies u latexu ---------
%----------------------------------------------------------------------------------------------------------------------
%-----------------------POPRAVLJA IMPLIES---------------------------
%-------------------------------------------------------------------
%ove commande dolje popravljaju \implies i \Longrightarrow , bolje je s ovime.
% Save original macros
% --------------------
%\usepackage{letltxmacro}

\let\OriginalLongrightarrow\Longrightarrow
\let\OriginalLongleftarrow\Longleftarrow

% Implement new macros
% --------------------
\usepackage{trimclip}
\DeclareRobustCommand\Longrightarrow{\NewRelbar\joinrel\Rightarrow}
\DeclareRobustCommand\Longleftarrow{\Leftarrow\joinrel\NewRelbar}

\makeatletter
\DeclareRobustCommand\NewRelbar{%
  \mathrel{%
    \mathpalette\@NewRelbar{}%
  }%
}
\newcommand*\@NewRelbar[2]{%
  % #1: math style
  % #2: unused
  \sbox0{$#1=$}%
  \sbox2{$#1\Rightarrow\m@th$}%
  \sbox4{$#1\Leftarrow\m@th$}%
  \clipbox{0pt 0pt \dimexpr(\wd2-.6\wd0) 0pt}{\copy2}%
  \kern-.2\wd0 %
  \clipbox{\dimexpr(\wd4-.6\wd0) 0pt 0pt 0pt}{\copy4}%
}
\makeatother
%-------------------------------------------------------------------
%-------------------------------------------------------------------
%-------------------------------------------------------------------
%----------------------------------------------------------------------------------------------------------------------
%======================================================================================================================


%======================================================================================================================
%------------------------- Ovdje sam namjestio da kada stavimo navodne znake, da izgleda kao na hrvatskom -------------
%----------------------------------------------------------------------------------------------------------------------
%za citiranje na hrvatskom
\usepackage{csquotes}
\MakeOuterQuote{"}
%----------------------------------------------------------------------------------------------------------------------
%======================================================================================================================


%======================================================================================================================
%------------------------- OVE NAREDBE ISPOD MJENJAJU KLASIČNI UNDERLINE U POBOLJŠANI ---------------------------------
%------------------------- (POBOLJŠANI UNDERLINE POŠTUJE PRELAMANJE RIJEČI NA KRAJU RETKA KADA NE STANE) --------------
%----------------------------------------------------------------------------------------------------------------------
%\NewCommandCopy{\oldunderline}{\underline}
\let\oldunderline\underline
\renewcommand{\underline}[1]{\ul{#1}}
%----------------------------------------------------------------------------------------------------------------------
%======================================================================================================================


%======================================================================================================================
%------------------------- Ovo su neke stvari od prije koje se vjerojatno mogu maknuti --------------------------------
%----------------------------------------------------------------------------------------------------------------------

%\usepackage[pdftex]{graphicx}
%\pagestyle{headings}

%NOTE: treba popraviti problem sto \vec{\mathbf r} ne postavlja strelicu na ispravnu lokaciju vec je malo slanted kao da je rijec o simbolu koji je slanted

%\edef\restoreparindent{\parindent=\the\parindent\relax}
%\usepackage[skip=\baselineskip]{parskip}
%\restoreparindent

%----------------------------------------------------------------------------------------------------------------------
%======================================================================================================================


%======================================================================================================================
%------------------------- Ovo mi treba za memoir paket da popravim margine -------------------------------------------
%----------------------------------------------------------------------------------------------------------------------

%\geometry{bindingoffset=0mm,nomarginpar,includeall,left=23mm,right=23mm,top=35mm,bottom=38mm}
 \geometry{bindingoffset=0mm,nomarginpar,includeall,left=19.55mm,right=19.55mm,top=29.75mm,bottom=32.3mm}
%\geometry{bindingoffset=0mm,nomarginpar,includeall,left=18.4mm,right=18.4mm,top=28mm,bottom=30.4mm}
%\geometry{bindingoffset=0mm,nomarginpar,includeall,left=16.1mm,right=16.1mm,top=24.5mm,bottom=26.6mm}
%\geometry{bindingoffset=0mm,nomarginpar,includeall,left=14.95mm,right=14.95mm,top=22.75mm,bottom=24.7mm}
%\geometry{bindingoffset=0mm,nomarginpar,includeall,left=13.8mm,right=13.8mm,top=21mm,bottom=22.8mm}

%\setlength\baselineskip{15pt} % ne znam sto je ovo, vjerojatno se isto moze maknuti
 
%----------------------------------------------------------------------------------------------------------------------
%======================================================================================================================


%======================================================================================================================
%------------------------- Ovdje ide spacing izmedju linija u naslovnoj stranici --------------------------------------
%----------------------------------------------------------------------------------------------------------------------

%ovo treba za title spacing

\setlength{\droptitle}{0pt}
\setlength{\dropBeforeTitle}{30mm}
\setlength{\dropAfterTitle}{10mm}

%----------------------------------------------------------------------------------------------------------------------
%======================================================================================================================


%======================================================================================================================
%------------------------- Jezik i stil za babelbib (bibliografiju) i valjda hyphenation ------------------------------
%------------------------- I jos reference za \cite i takve stvari valjda isto to tu sve ide --------------------------
%----------------------------------------------------------------------------------------------------------------------

%\usepackage[languagenames,fixlanguage,croatian]{babelbib} %zahtjeva datotetku croatian.bdf
\usepackage[languagenames,fixlanguage,english]{babelbib}

\bibliographystyle{babplain} % babamspl ili babplain

\usepackage[pdftex, hyperfootnotes=false]{hyperref}

%----------------------------------------------------------------------------------------------------------------------
%======================================================================================================================


%======================================================================================================================
%------------------------- Ovdje definiramo svoje operatore -----------------------------------------------------------
%----------------------------------------------------------------------------------------------------------------------

\DeclareMathOperator{\tg}{tg}
\DeclareMathOperator{\ctg}{ctg}
\DeclareMathOperator{\arctg}{arctg}
\DeclareMathOperator{\arcctg}{arcctg}
\DeclareMathOperator{\sh}{sh}
\DeclareMathOperator{\ch}{ch}
\DeclareMathOperator{\tgh}{th}
\DeclareMathOperator{\cth}{cth}
\DeclareMathOperator{\Ker}{Ker}

\DeclareMathOperator{\Ext}{Ext}
\DeclareMathOperator{\Int}{Int}
\DeclareMathOperator{\diag}{diag}
\DeclareMathOperator{\Span}{span}
\DeclareMathOperator{\trag}{trag}
\DeclareMathOperator{\Ric}{Ric}
\DeclareMathOperator{\fff}{I}
\DeclareMathOperator{\sff}{II}
\DeclareMathOperator{\sgn}{sgn}
\DeclareMathOperator{\const}{const.}
\DeclareMathOperator{\id}{id}
\DeclareMathOperator{\supp}{supp}
\DeclareMathOperator{\grad}{grad}
\DeclareMathOperator{\rg}{rg}
\DeclareMathOperator{\End}{End}
\DeclareMathOperator{\trace}{trag}
\DeclareMathOperator{\Sym}{Sym}
\DeclareMathOperator{\Alt}{Alt}
\DeclareMathOperator{\firstfundform}{I}
\DeclareMathOperator{\secondfundform}{II}

\let\Im\relax
\DeclareMathOperator{\Im}{Im}
\DeclareMathOperator{\Aut}{Aut}

\DeclareMathOperator{\separator}{|}

%----------------------------------------------------------------------------------------------------------------------
%======================================================================================================================


%======================================================================================================================
%------------------------- Ovdje definiram svoje naredbe --------------------------------------------------------------
%----------------------------------------------------------------------------------------------------------------------
%NOTE: treba popraviti problem sto \vec{\mathbf r} ne postavlja strelicu na ispravnu lokaciju vec je malo slanted kao da je rijec o simbolu koji je slanted

\newcommand{\vectorstyle}[1]{\vec{\mathbf{#1}}}
\newcommand{\vectorderivative}{{\,\prime}}

%-------------------------------------------------------------------------------------------------------------------------------

\newcommand{\norm}{|\kern-0.083em|\kern-0.083em|}
\newcommand{\transp}{\textsc{t}}
\newcommand{\euklidski}{\textsc{e}}
\newcommand{\minkowski}{\textsc{m}}
\newcommand{\ArrayOffset}{$\kern-0.76016em$ }
\newcommand{\ArrayAlignOffset}{\kern-0.21em }
\newcommand{\newoverbrace}[1]{\aoverbrace[L1R]{#1}[U]}
\newcommand{\newunderbrace}[1]{\aunderbrace[l1r]{#1}[D]}
\newcommand{\prescript}[2]{\vphantom{#2}^{#1}#2}
% ovo ne radi nesto, gledao sam na stackechangeu, evo link
% https://tex.stackexchange.com/questions/272850/use-verbatim-inside-newcommand
% \newcommand{\inlinecode}[1]{\verb|#1|}

\def\derivative{\dot}
\def\dderivative{\ddot}
\def\ddderivative{\dddot}
%\def\derivative{\overset{\text{.}}}
%\def\dderivative{\overset{\text{..}}}

\newcommand{\qbinom}[2]{\binom{#1}{#2}_q}
\newcommand{\qint}[1]{[#1]_q}
\newcommand{\qfallingpower}[1]{{\oldunderline{#1}}_q}
\newcommand{\qrisingpower}[1]{{\overline{#1}}_q}
%----------------------------------------------------------------------------------------------------------------------
%======================================================================================================================


%======================================================================================================================
%------------------------- Ovdje definiram svoje naredbe za uredjivanje teksta, tipa \emph ali nesto svoje ------------
%----------------------------------------------------------------------------------------------------------------------
\newcommand{\alert}[1]{{\color{red}#1}}
\newcommand{\hide}[1]{{\color{gray}#1}}

%----------------------------------------------------------------------------------------------------------------------
%======================================================================================================================

