%%%%%%%%%%%%%%%%%%%%%%%%%%%%%%%%%%%%%%%%%%%%%%%%%%%%%%%%%%%%%%%%%%%%%%%%%%%%%%%%%%%%%%%%%%%%%%%%%%%%%%%%%%%%%%%%%%%%%%%
%%%%%%%%%%%%%%%%%%%%%%%%%%%%%%%%%%%%%%%%%%%%%%%%%%%%%%%%%%%%%%%%%%%%%%%%%%%%%%%%%%%%%%%%%%%%%%%%%%%%%%%%%%%%%%%%%%%%%%%
%%%%%%%%%%%%%%%%                                                                                          %%%%%%%%%%%%%
%%%%%%%%%%%%%%%% Settingsi su na dnu, tipa margine i tako sto, al ostalo se isto moze urediti             %%%%%%%%%%%%%
%%%%%%%%%%%%%%%%                                                                                          %%%%%%%%%%%%%
%%%%%%%%%%%%%%%%%%%%%%%%%%%%%%%%%%%%%%%%%%%%%%%%%%%%%%%%%%%%%%%%%%%%%%%%%%%%%%%%%%%%%%%%%%%%%%%%%%%%%%%%%%%%%%%%%%%%%%%
%%%%%%%%%%%%%%%%%%%%%%%%%%%%%%%%%%%%%%%%%%%%%%%%%%%%%%%%%%%%%%%%%%%%%%%%%%%%%%%%%%%%%%%%%%%%%%%%%%%%%%%%%%%%%%%%%%%%%%%


%======================================================================================================================
%------------------------- Ovdje stvari koje su bile u prezentaciji ali nisam iskoristion, zakomentirano --------------
%----------------------------------------------------------------------------------------------------------------------
%\usepackage{lmodern}
%\usepackage[T1]{fontenc}
%\usepackage[utf8]{inputenc}

%----------------------------------------------------------------------------------------------------------------------
%======================================================================================================================


%======================================================================================================================
%------------------------- Ovdje kopiram stvari koje su mi bile iz .sty filea, ali zaprvo -----------------------------
%------------------------- se moze vjerojatno maknuti svasta od toga --------------------------------------------------
%----------------------------------------------------------------------------------------------------------------------

\usepackage[croatian]{babel}
%\usepackage[dvipsnames]{xcolor}
\usepackage{amsmath}
\usepackage{amssymb}
\usepackage{wasysym}
\usepackage{textcomp}
\usepackage{array}
\usepackage{arcs}
\usepackage{polynom}
\usepackage{cancel}
\usepackage{enumerate}
\usepackage{dcolumn}
\usepackage{xlop}
\usepackage{geometry}
\usepackage{xcolor}

\usepackage{amsthm}

% OVO JE IZ STILA ZA DIPLOMSKI KOJI ODREDJUJE MARGINE I TAKO TO
%\geometry{bindingoffset=12mm,nomarginpar,includeall,left=23mm,right=23mm,top=35mm,bottom=38mm}
%\setlength\baselineskip{15pt}

%----------------------------------------------------------------------------------------------------------------------
%PAKETI KOJE SAM JA DODAO KOJI NISU STAJALI U ORIGINALNOM PREDLOŠKU ZA DIPLOMSKI

\usepackage{lipsum}
%MORA SE MAKNUTI ENUMITEM ZA beamer PREZENTACIJU
%\usepackage{enumitem}				%za enumeraciju itemiziranje
\usepackage{soul}					%za podcrtavanje
\usepackage{soulutf8}				%za podrcrtavanje
\usepackage{centernot}				%za negacjiu implikacije \centernot\implies da prekrizeno bude na crti
\usepackage{abraces}				%za lijepši overbrace i underbrace
\usepackage{import}                 %za lijepo importanje figuresa iz inkscapea

%----------------------------------------------------------------------------------------------------------------------
%======================================================================================================================


%======================================================================================================================
%------------------------- Ovdje namjestam stil za definiciju, teorem, propozicije i sve to ... -----------------------
%----------------------------------------------------------------------------------------------------------------------

%----------------------------------------------------------------------------------------------------------------------
%----------------OVAJ JEDAN REDAK ISPOD SAM JA DODAO JER MI JE TAKO LJEPŠE RADITI DIPLOMSKI, MAKNUTI PO ŽELJI----------
\theoremstyle{definition}
%----------------------------------------------------------------------------------------------------------------------
%----------------------------------------------------------------------------------------------------------------------
%----------------------------------------------------------------------------------------------------------------------

%\newtheorem{theorem}{Teorem}[\enumeration] % ovo je neki stari pokusaj, ali nije proslo

%ako zelimo enumeraciju da nije 0.0.1 (kada stavis zadatak prije chaptera),
%onda treba u documentsettings.tex zakomentirati
%liniju \newtheorem{theorem}{Teorem}[section]
%i odkomentirati prikladnu liniju
%\newtheorem{theorem}{Teorem}
%\newtheorem{theorem}{Teorem}[chapter]
%\newtheorem{theorem}{Teorem}[section]

%ovo je sada novo s \enumeration commandom u
%\enumeration
%\newtheorem{theorem}{Teorem}[]
%\newtheorem{lemma}[theorem]{Lema}
%\newtheorem{corollary}[theorem]{Korolar}
%\newtheorem{definition}[theorem]{Definicija}
%\newtheorem{remark}[theorem]{Napomena}
%\newtheorem{proposition}[theorem]{Propozicija}
%\newtheorem{example}[theorem]{Primjer}
%\newtheorem{exercise}[theorem]{Zadatak}

%----------------------------------------------------------------------------------------------------------------------
%======================================================================================================================


%======================================================================================================================
%------------------------- Ovdje sam iskopirao kod s stackexchangea koji mi uredjuje proof style ----------------------
%----------------------------------------------------------------------------------------------------------------------
%-------------------------------------------------------------------
%----------------------REDEF PROOF ENVIRONMENT----------------------
%-------------------------------------------------------------------
%ove komande mjenjaju proof environment u onakav kakav se meni više sviđa, po volji maknuti

\expandafter\let\expandafter\oldproof\csname\string\proof\endcsname
\let\oldendproof\endproof
\renewenvironment{proof}[1][dokaz]{%
  \oldproof[\underline{\textsc{#1}}\nopunct]%
}{\oldendproof}
%-------------------------------------------------------------------
%------------------------ALTERNATIVNO-------------------------------
%-------------------------------------------------------------------
%\newenvironment{myproof}[1][dokaz]{%
%  \proof[\underline{\textsc{#1}}\nopunct]%
%}{\endproof}
%-------------------------------------------------------------------
%-------------------------------------------------------------------
%-------------------------------------------------------------------
%----------------------------------------------------------------------------------------------------------------------
%======================================================================================================================


%======================================================================================================================
%----------------------------------------------------------------------------------------------------------------------
%\hyphenation{di-men-zi-o-nal-nom} NOTE: ovo ne radi iz nekog razloga

%\addto\croatian{\renewcommand\proofname{ABC}}
%\renewcommand*{\proofname}{\textsc{dokaz}}

%\usepackage{ulem}
%mijenja \emph u \underline sa line-breakom
%alternativno, koristiti soul package
%----------------------------------------------------------------------------------------------------------------------
%======================================================================================================================


%======================================================================================================================
%------------------------- Ovdje sam onda isto tako sklepao nesto za solution environment -----------------------------
%----------------------------------------------------------------------------------------------------------------------
%----------------------------------------------------------------------------------------------------------------------
%----------------------------------------------------------------------------------------------------------------------
%----------------------------------------------------------------------------------------------------------------------

%---------DEFINICIJA SOLUTION ENVIRONMENTA--------------------
%\newenvironment{solution}[1][rje\v senje]{%
%  \oldproof[\underline{\textsc{#1}}\nopunct]%
%}{\oldendproof}

%----------------------------------------------------------------------------------------------------------------------
%======================================================================================================================


%======================================================================================================================
%------------------------- Ovdje sam iskopirao niz komandi s stackexchangea koji popravljaju \impies u latexu ---------
%----------------------------------------------------------------------------------------------------------------------
%-----------------------POPRAVLJA IMPLIES---------------------------
%-------------------------------------------------------------------
%ove commande dolje popravljaju \implies i \Longrightarrow , bolje je s ovime.
% Save original macros
% --------------------
%\usepackage{letltxmacro}

\let\OriginalLongrightarrow\Longrightarrow
\let\OriginalLongleftarrow\Longleftarrow

% Implement new macros
% --------------------
\usepackage{trimclip}
\DeclareRobustCommand\Longrightarrow{\NewRelbar\joinrel\Rightarrow}
\DeclareRobustCommand\Longleftarrow{\Leftarrow\joinrel\NewRelbar}

\makeatletter
\DeclareRobustCommand\NewRelbar{%
  \mathrel{%
    \mathpalette\@NewRelbar{}%
  }%
}
\newcommand*\@NewRelbar[2]{%
  % #1: math style
  % #2: unused
  \sbox0{$#1=$}%
  \sbox2{$#1\Rightarrow\m@th$}%
  \sbox4{$#1\Leftarrow\m@th$}%
  \clipbox{0pt 0pt \dimexpr(\wd2-.6\wd0) 0pt}{\copy2}%
  \kern-.2\wd0 %
  \clipbox{\dimexpr(\wd4-.6\wd0) 0pt 0pt 0pt}{\copy4}%
}
\makeatother
%-------------------------------------------------------------------
%-------------------------------------------------------------------
%-------------------------------------------------------------------
%----------------------------------------------------------------------------------------------------------------------
%======================================================================================================================


%======================================================================================================================
%------------------------- Ovdje sam namjestio da kada stavimo navodne znake, da izgleda kao na hrvatskom -------------
%----------------------------------------------------------------------------------------------------------------------
%za citiranje na hrvatskom
\usepackage{csquotes}
\MakeOuterQuote{"}
%----------------------------------------------------------------------------------------------------------------------
%======================================================================================================================


%======================================================================================================================
%------------------------- OVE NAREDBE ISPOD MJENJAJU KLASIČNI UNDERLINE U POBOLJŠANI ---------------------------------
%------------------------- (POBOLJŠANI UNDERLINE POŠTUJE PRELAMANJE RIJEČI NA KRAJU RETKA KADA NE STANE) --------------
%----------------------------------------------------------------------------------------------------------------------
%\NewCommandCopy{\oldunderline}{\underline}
\let\oldunderline\underline
\renewcommand{\underline}[1]{\ul{#1}}
%----------------------------------------------------------------------------------------------------------------------
%======================================================================================================================


%======================================================================================================================
%------------------------- Ovo su neke stvari od prije koje se vjerojatno mogu maknuti --------------------------------
%----------------------------------------------------------------------------------------------------------------------

%\usepackage[pdftex]{graphicx}
%\pagestyle{headings}

%NOTE: treba popraviti problem sto \vec{\mathbf r} ne postavlja strelicu na ispravnu lokaciju vec je malo slanted kao da je rijec o simbolu koji je slanted

%\edef\restoreparindent{\parindent=\the\parindent\relax}
%\usepackage[skip=\baselineskip]{parskip}
%\restoreparindent

%----------------------------------------------------------------------------------------------------------------------
%======================================================================================================================


%======================================================================================================================
%------------------------- Jezik i stil za babelbib (bibliografiju) i valjda hyphenation ------------------------------
%------------------------- I jos reference za \cite i takve stvari valjda isto to tu sve ide --------------------------
%----------------------------------------------------------------------------------------------------------------------

%\usepackage[languagenames,fixlanguage,croatian]{babelbib} %zahtjeva datotetku croatian.bdf
\usepackage[languagenames,fixlanguage,english]{babelbib}

%\bibliographystyle{babplain} % babamspl ili babplain

%\usepackage[pdftex, hyperfootnotes=false]{hyperref}
%   hyperref je po defaultu ukljucen u beamer document class, pa ako covjek zeli neke settingse, nominalno
%   [pdftex, hyperfootnote=false], onda to treba kopirati u settingse documentclassa, nominalno:
%   \documentclass[aspectratio=43, hyperref={pdftex, hyperfootnotes=false}]{beamer}

%----------------------------------------------------------------------------------------------------------------------
%======================================================================================================================


%======================================================================================================================
%------------------------- Ovdje definiramo svoje operatore -----------------------------------------------------------
%----------------------------------------------------------------------------------------------------------------------

\DeclareMathOperator{\tg}{tg}
\DeclareMathOperator{\ctg}{ctg}
\DeclareMathOperator{\arctg}{arctg}
\DeclareMathOperator{\arcctg}{arcctg}
\DeclareMathOperator{\sh}{sh}
\DeclareMathOperator{\ch}{ch}
\DeclareMathOperator{\tgh}{th}
\DeclareMathOperator{\cth}{cth}
\DeclareMathOperator{\Ker}{Ker}

\DeclareMathOperator{\Ext}{Ext}
\DeclareMathOperator{\Int}{Int}
\DeclareMathOperator{\diag}{diag}
\DeclareMathOperator{\Span}{span}
\DeclareMathOperator{\trag}{trag}
\DeclareMathOperator{\Ric}{Ric}
\DeclareMathOperator{\sgn}{sgn}
\DeclareMathOperator{\const}{const.}
\DeclareMathOperator{\id}{id}
\DeclareMathOperator{\supp}{supp}
\DeclareMathOperator{\grad}{grad}
\DeclareMathOperator{\rg}{rg}
\DeclareMathOperator{\End}{End}
\DeclareMathOperator{\trace}{trag}
\DeclareMathOperator{\Sym}{Sym}
\DeclareMathOperator{\Alt}{Alt}
\DeclareMathOperator{\fff}{I}
\DeclareMathOperator{\sff}{II}
\DeclareMathOperator{\firstfundform}{I}
\DeclareMathOperator{\secondfundform}{II}

\let\Im\relax
\DeclareMathOperator{\Im}{Im}
\DeclareMathOperator{\Aut}{Aut}

\DeclareMathOperator{\separator}{|}

%----------------------------------------------------------------------------------------------------------------------
%======================================================================================================================


%======================================================================================================================
%------------------------- Ovdje definiram svoje naredbe --------------------------------------------------------------
%----------------------------------------------------------------------------------------------------------------------
%NOTE: treba popraviti problem sto \vec{\mathbf r} ne postavlja strelicu na ispravnu lokaciju vec je malo slanted kao da je rijec o simbolu koji je slanted

\newcommand{\vectorstyle}[1]{\vec{\mathbf{#1}}}
\newcommand{\vectorderivative}{{\,\prime}}

%-------------------------------------------------------------------------------------------------------------------------------

\newcommand{\norm}{|\kern-0.083em|\kern-0.083em|}
\newcommand{\transp}{{\textsc{t}}}
\newcommand{\euklidski}{\textsc{e}}
\newcommand{\minkowski}{\textsc{m}}
\newcommand{\ArrayOffset}{$\kern-0.76016em$ }
\newcommand{\ArrayAlignOffset}{\kern-0.21em }
\newcommand{\newoverbrace}[1]{\aoverbrace[L1R]{#1}[U]}
\newcommand{\newunderbrace}[1]{\aunderbrace[l1r]{#1}[D]}
\newcommand{\prescript}[2]{\vphantom{#2}^{#1}#2}
% ovo ne radi nesto, gledao sam na stackechangeu, evo link
% https://tex.stackexchange.com/questions/272850/use-verbatim-inside-newcommand
% \newcommand{\inlinecode}[1]{\verb|#1|}

\def\derivative{\dot}
\def\dderivative{\ddot}
\def\ddderivative{\dddot}
%\def\derivative{\overset{\text{.}}}
%\def\dderivative{\overset{\text{..}}}

\newcommand{\qbinom}[2]{\binom{#1}{#2}_q}
\newcommand{\qint}[1]{[#1]_q}
\newcommand{\qfallingpower}[1]{{\oldunderline{#1}}_q}
\newcommand{\qrisingpower}[1]{{\overline{#1}}_q}
%----------------------------------------------------------------------------------------------------------------------
%======================================================================================================================


%======================================================================================================================
%------------------------- Ovdje definiram svoje naredbe za uredjivanje teksta, tipa \emph ali nesto svoje ------------
%----------------------------------------------------------------------------------------------------------------------
%\newcommand{\alert}[1]{{\color{red}#1}}
\newcommand{\hide}[1]{{\color{gray}#1}}

%----------------------------------------------------------------------------------------------------------------------
%======================================================================================================================

%======================================================================================================================
%------------------------- Ovdje idu sada stvari iz metropolis teme za prezentaciju -----------------------------------
%----------------------------------------------------------------------------------------------------------------------
%\usetheme[progressbar=frametitle]{metropolis}
\usetheme{metropolis}
\setbeamertemplate{frame numbering}[fraction]
\useoutertheme{metropolis}
\useinnertheme{metropolis}
\usefonttheme{metropolis}
%\usecolortheme{spruce}
%\setbeamercolor{background canvas}{bg=white}
\usefonttheme{serif}

%\definecolor{mygreen}{rgb}{.125,.5,.25}
%\usecolortheme[named=mygreen]{structure}

% Copyright 2012 by Alan Munn <amunn@msu.edu>  Modified for MSU colours 
% Copyright 2004 by Madhusudan Singh <madhusudan.singh@gmail.com>
%
% This file may be distributed and/or modified
%
% 1. under the LaTeX Project Public License and/or
% 2. under the GNU Public License.
%
% See the file doc/licenses/LICENSE for more details.
%

\mode<presentation>

\definecolor{MyColor}{RGB}{204,0,0}
\definecolor{MyBgColor}{RGB}{219,76,76}
\definecolor{MyFontColor}{RGB}{50,0,0} 
% Approximate match for Michigan State University; official colour appears too dark

\setbeamercolor{normal text}{fg=black, bg=white}
\setbeamercolor{example text}{fg=black}

\setbeamercolor{alerted text}{fg=MyColor!80!white}
\setbeamercolor*{palette primary}{fg=MyColor!50!black,bg=white!90!MyBgColor}
\setbeamercolor*{palette secondary}{fg=MyColor!60!black,bg=white!60!MyColor}
\setbeamercolor*{palette tertiary}{bg=MyColor!70!black,fg=white!50!MyColor}
\setbeamercolor*{palette quaternary}{fg=MyColor!80!black,bg=white!20!MyColor}

\setbeamercolor*{sidebar}{fg=MyColor,bg=MyColor!75!white}

\setbeamercolor*{palette sidebar primary}{fg=MyColor!10!black}
\setbeamercolor*{palette sidebar secondary}{fg=white}
\setbeamercolor*{palette sidebar tertiary}{fg=MyColor!50!black}
\setbeamercolor*{palette sidebar quaternary}{fg=white!10!MyColor}

\setbeamercolor*{titlelike}{parent=palette primary}
\setbeamercolor{frametitle}{bg=white!80!MyBgColor}
\setbeamercolor{frametitle right}{bg=white!60!MyColor}

\setbeamercolor*{separation line}{}
\setbeamercolor*{fine separation line}{}
\setbeamercolor{block body}{parent=normal text,use=block title,bg=red!0,fg=}
\mode
<all>

%----------------------------------------------------------------------------------------------------------------------
%======================================================================================================================


%======================================================================================================================
%------------------------- Ovdje ide definiranje naslovne stranice i jos neki dodatni settingsi -----------------------
%----------------------------------------------------------------------------------------------------------------------
\newcommand{\presentationdate}[1]{\def\myPresentationDate{#1}}
\newcommand{\authorname}[1]{\author{\texorpdfstring{#1\hfill \myPresentationDate{}}{#1}}}
%\subtitle{Seminar za geometriju}
%\subtitle{Subtitle Here}
\newcommand{\institutename}[1]{
    \institute{#1%
            \begin{figure}%
            \centering%
            \includegraphics[width=0.5\textwidth]{assets/fer.png}%
            \quad%
            \includegraphics[width=0.3\textwidth]{assets/hrzz.jpg}%
            \end{figure}%
            \hfill rad podr\v zan HRZZ projektom IP-2020-02-9752%
    }%
}
%\institute{\large \textbf{Learning Outcomes}: \\[6pt] Identify properties of elementary functions (formed by composition of power, exponential, logarithmic, and trigonometric functions and their inverses).}
%\date{\scriptsize{Zagreb, srpanj 2021.}}
\date{}

\setbeamercovered{transparent=5}

\metroset{block=fill}

% Ovdje sada kopiram staru verziju ovog setupa cisto tako da ako se zelim vratiti na staro:
%   \title{Konstrukcija dizajna metodom takti\v cke dekompozicije}
%   %\subtitle{Seminar za geometriju}
%   %\subtitle{Subtitle Here}
%   \author{\texorpdfstring{Filip Martinović\hfill velja\v ca, 2023.}{Filip Martinović}}
%   \institute[Sveučilište u Zagrebu]{Sveučilište u Zagrebu, Fakultet elektrotehnike i ra\v cunarstva
%           \begin{figure}
%           \centering
%           \includegraphics[width=0.5\textwidth]{slike/unilogo.png}
%           \quad
%           \includegraphics[width=0.3\textwidth]{slike/hrzz.jpg}
%           \end{figure}
%           \hfill rad podr\v zan HRZZ projektom IP-2020-02-9752
%   }
%   %\institute{\large \textbf{Learning Outcomes}: \\[6pt] Identify properties of elementary functions (formed by composition of power, exponential, logarithmic, and trigonometric functions and their inverses).}
%   %\date{\scriptsize{Zagreb, srpanj 2021.}}
%   \date{}
%   
%   \setbeamercovered{transparent=5}
%   
%   \metroset{block=fill}

\setbeamertemplate{frametitle continuation}{} % ovo sluzi da kada napravis allowframebreaks, da ne numerira to

%----------------------------------------------------------------------------------------------------------------------
%======================================================================================================================

